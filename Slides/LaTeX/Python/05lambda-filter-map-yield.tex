\documentclass[serif, aspectratio=169]{beamer}
\usepackage[T1]{fontenc} 
\usepackage{fourier}
\usepackage{hyperref}
\usepackage{latexsym,amsmath,xcolor,multicol,booktabs,calligra}
\usepackage{graphicx,pstricks,listings,stackengine}
\usepackage{listings}

\author{Dr.Hajialiasgari}
\title{Introduction To Python}
\institute{
    Tehran University \\
    Of\\
    Medical Science
}
\date{\small \today}
\usepackage{UoWstyle}

% Define custom colors and styles for listings
\definecolor{deepblue}{rgb}{0,0,0.5}
\definecolor{deepred}{RGB}{153,0,0}
\definecolor{deepgreen}{rgb}{0,0.5,0}
\definecolor{halfgray}{gray}{0.55}

\lstset{
    basicstyle=\ttfamily\small,
    keywordstyle=\bfseries\color{deepblue},
    emphstyle=\ttfamily\color{deepred},
    stringstyle=\color{deepgreen},
    numbers=left,
    numberstyle=\small\color{halfgray},
    rulesepcolor=\color{red!20!green!20!blue!20},
    frame=shadowbox,
}

\begin{document}

\begin{frame}
    \titlepage
    \vspace*{-0.6cm}
    \begin{figure}[htpb]
        \begin{center}
            \includegraphics[keepaspectratio, scale=0.05]{Tumsl-logo.png}
        \end{center}
    \end{figure}
\end{frame}

\begin{frame}    
\tableofcontents[sectionstyle=show, subsectionstyle=show/shaded/hide, subsubsectionstyle=show/shaded/hide]
\end{frame}

\section{Filter ,Map ,Lambda ,Yield}

\begin{frame}{Lambda Functions}
    \begin{itemize}
        \item A lambda function is a small, anonymous function that can have any number of arguments but only one expression. It's often used for short, simple operations where defining a full function might be unnecessary.
        \item \texttt{\color{red}Characteristics:}
        \item No def keyword or name.
        \item Limited to one expression (e.g., no loops or multiple statements).
        \item Typically used in combination with functions like filter, map, and sorted.
    \end{itemize}
\end{frame}

\begin{frame}[fragile]{lambda example}
    \begin{lstlisting}
# Regular function
def multiply(x, y):
    return x * y

# Equivalent lambda function
multiply = lambda x, y: x * y

# Usage
result = multiply(4, 5) 
# Output: 20
   \end{lstlisting}
\end{frame}

\begin{frame}{Filter}
    \begin{itemize}
        \item The \texttt{\color{red}filter()} function applies a filtering condition to an iterable (e.g., list, tuple) and returns only the elements that satisfy the condition.
        \item Returns: A filter object, which can be converted to a list or other iterable types.
        \item Use Cases: Extracting specific elements based on a criterion (e.g., filtering even numbers, removing None values).
    \end{itemize}
\end{frame}

\begin{frame}[fragile]{filter Example}
	\begin{lstlisting}
# Remove strings with less than 5 characters
words = ["apple", "kiwi", "banana", "pear"]
filtered_words = filter(lambda word: len(word) >= 5, words)

print(list(filtered_words))  
# Output: ['apple', 'banana']
	\end{lstlisting}
\end{frame}

\begin{frame}{Map}
    \begin{itemize}
        \item The \texttt{\color{red}map()} function applies a transformation function to all elements in an iterable. It’s often used when you need to perform the same operation on each element of a sequence.
        \item Returns: A map object, which can be converted to a list or other iterable types.
        \item Use Cases: Converting data types, performing mathematical operations, and applying transformations.
    \end{itemize}
\end{frame}

\begin{frame}[fragile]{map Example}
    \begin{lstlisting}
# Convert Celsius to Fahrenheit
temps_celsius = [0, 20, 30, 40]
temps_fahrenheit = map(lambda c: (c * 9/5) + 32, temps_celsius)

print(list(temps_fahrenheit)) 
# Output: [32.0, 68.0, 86.0, 104.0]
    \end{lstlisting}
\end{frame}

\begin{frame}{Generator Functions (yield)}
    \begin{itemize}
        \item Generators are a type of iterable, like lists, but they produce items lazily (one at a time). This makes them memory-efficient, especially for large datasets.

        \item \texttt{\color{red}Key Features:}
        \item Use yield to produce a value and pause execution.
        \item Can be resumed to continue producing more values.
        \item Automatically implements the iterator protocol.
    \end{itemize}
\end{frame}

\begin{frame}{Advantages}
    \begin{itemize}
        \item Memory Efficient: Stores only the current state, not all values.
        \item Lazy Evaluation: Generates values only when needed.
        \item Infinite Sequences: Ideal for generating infinite sequences like Fibonacci numbers or prime numbers.
    \end{itemize}
\end{frame}


\begin{frame}{Comparison to Regular Functions:}
    \begin{itemize}
         \item A regular function returns all values at once and terminates.
         \item A generator function yields values one at a time, preserving memory.
    \end{itemize}

\end{frame}
\begin{frame}[fragile]{yield Example}
    \begin{lstlisting}
# Generate Fibonacci sequence up to a limit
def fibonacci(limit):
    a, b = 0, 1
    while a < limit:
        yield a
        a, b = b, a + b

# Using the generator
for num in fibonacci(10):
    print(num)
# Output: 0, 1, 1, 2, 3, 5, 8
    \end{lstlisting}
\end{frame}

\begin{frame}
    \begin{center}
        {\Huge\ End of Python Course !}
    \end{center}
\end{frame}

\end{document}

