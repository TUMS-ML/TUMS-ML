\documentclass[serif, aspectratio=169]{beamer}
%\documentclass[serif]{beamer}  % for 4:3 ratio
\usepackage[T1]{fontenc} 
\usepackage{fourier}
\usepackage{hyperref}
\usepackage{latexsym,amsmath,xcolor,multicol,booktabs,calligra}
\usepackage{graphicx,pstricks,listings,stackengine}
\usepackage{listings}

\author{Dr.Hajialiasgari}
\title{Introduction To Python}
\institute{
    Tehran University \\
    Of\\
    Medical Science
}
\date{\small \today}
\usepackage{UoWstyle}

% defs
\def\cmd#1{\texttt{\color{red}\footnotesize $\backslash$#1}}
\def\env#1{\texttt{\color{blue}\footnotesize #1}}
\definecolor{deepblue}{rgb}{0,0,0.5}
\definecolor{deepred}{RGB}{153,0,0}
\definecolor{deepgreen}{rgb}{0,0.5,0}
\definecolor{halfgray}{gray}{0.55}

\lstset{
    basicstyle=\ttfamily\small,
    keywordstyle=\bfseries\color{deepblue},
    emphstyle=\ttfamily\color{deepred},
    stringstyle=\color{deepgreen},
    numbers=left,
    numberstyle=\small\color{halfgray},
    rulesepcolor=\color{red!20!green!20!blue!20},
    frame=shadowbox,
}

\begin{document}

\begin{frame}
    \titlepage
    \vspace*{-0.6cm}
    \begin{figure}[htpb]
        \begin{center}
            \includegraphics[keepaspectratio, scale=0.05]{Tumsl-logo.png}
        \end{center}
    \end{figure}
\end{frame}

\begin{frame}    
\tableofcontents[sectionstyle=show, subsectionstyle=show/shaded/hide, subsubsectionstyle=show/shaded/hide]
\end{frame}

\section{Functions}

\begin{frame}{Python Functions}
    \begin{itemize}
        \item There are two kinds of functions in Python:
        \begin{itemize}
            \item \textbf{Built-in functions} provided as part of Python, such as \texttt{print()}, \texttt{input()}, \texttt{type()}, etc.
            \item \textbf{User-defined functions} which are created by the programmer.
        \end{itemize}
        \item Function names are treated as "new reserved words."
    \end{itemize}
\end{frame}

\begin{frame}{Function Definition}
    \begin{itemize}
        \item A function is reusable code that takes arguments, performs a computation, and returns a result.
        \item Defined using the \texttt{def} keyword.
    \end{itemize}
\end{frame}

\section{Build Your Function}

\begin{frame}[fragile]{Build Your Function}
    \begin{itemize}
        \item Functions are defined with the \texttt{def} keyword and optional parameters in parentheses.
        \item Example:
    \end{itemize}
    \begin{lstlisting}
def greet(name):
    return "Hello, " + name + "!"
    \end{lstlisting}
\end{frame}

\section{Definition and Uses}

\begin{frame}{Definition and Uses}
    \begin{itemize}
        \item After defining a function, it can be invoked multiple times.
        \item This is known as the "store and reuse" pattern.
    \end{itemize}
\end{frame}

\section{Arguments}

\begin{frame}[fragile]{Arguments}
    \begin{itemize}
        \item Arguments are values passed to a function.
        \item Example:
    \end{itemize}
    \begin{lstlisting}
largest = max(3, 7, 2, 5)
print(largest)  # Output: 7
    \end{lstlisting}
\end{frame}

\section{Parameter}

\begin{frame}[fragile]{Parameter}
    \begin{itemize}
        \item A parameter is a variable in the function definition that refers to an argument during function calls.
        \item Example:
    \end{itemize}
    \begin{lstlisting}
def square(number):
    return number * number
    \end{lstlisting}
\end{frame}

\section{Return Values}

\begin{frame}[fragile]{Return Values}
    \begin{itemize}
        \item \texttt{return} sends a value back to the caller.
        \item Example:
    \end{itemize}
    \begin{lstlisting}
def add(a, b):
    return a + b
    \end{lstlisting}
\end{frame}

\section{Multiple Parameters / Arguments}

\begin{frame}[fragile]{Multiple Parameters / Arguments}
    \begin{itemize}
        \item Functions can take multiple parameters.
        \item Example:
    \end{itemize}
    \begin{lstlisting}
def multiply(x, y):
    return x * y
    \end{lstlisting}
\end{frame}

\section{Void (Non-Fruitful) Functions}

\begin{frame}[fragile]{Void (Non-Fruitful) Functions}
    \begin{itemize}
        \item Void functions do not return a value.
        \item Example:
    \end{itemize}
    \begin{lstlisting}
def greet(name):
    print("Hello, " + name + "!")
    \end{lstlisting}
\end{frame}

\section{Advantages}
\begin{frame}{Advantages}
    \begin{itemize}
        \item Organize code into logical chunks.
        \item Avoid repetition by reusing code.
    \end{itemize}
\end{frame}

\begin{frame}
    \begin{center}
        {\Huge End of Functions}
    \end{center}
\end{frame}

\end{document}
